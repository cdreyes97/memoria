\secnumbersection{VALIDACIÓN DE LA SOLUCIÓN}

Se debe validar la solución propuesta. Esto significa probar o demostrar que la solución propuesta es válida para el entorno donde fue planteada.

Tradicionalmente es una etapa crítica, pues debe comprobarse por algún medio que vuestra propuesta es básicamente válida. En el caso de un desarrollo de software es la construcción y sus pruebas; en el caso de propuestas de modelos, guías o metodologías podrían ser desde la aplicación a un caso real hasta encuestas o entrevistas con especialistas; en el caso de mejoras de procesos u optimizaciones, podría ser comparar la situación actual (previa a la memoria) con la situación final (cuando la memoria está ya implementada) en base a un conjunto cuantitativo de indicadores o criterios.

\subsection{EJEMPLO DE COMO CITAR TABLAS}

Se colocó una tabla que se puede referenciar también desde el texto (Ver tabla \ref{table:coloquios}).

\begin{table}[h]
    \centering
    \caption{\label{table:coloquios} Coloquios del Ciclo de Charlas Informática.} Fuente: Elaboración Propia.
    \begin{tabular}{|p{7cm}|p{7cm}|}
        \hline
        Título Coloquio & Presentador, País \\
        \hline
        ``Sensible, invisible, sometimes tolerant, heterogeneous, decentralized and interoperable... and we still need to assure its quality...''' & Guilherme Horta Travassos, Brasil.\\
        \hline
        ``Dispersed Multiphase Flow Modeling: From Environmental to Industrial Applications''' & Orlando Ayala, EE.UU.\\
        \hline
        ``Líneas de Producto Software Dinámicas para Sistemas atentos el Contexto''' & Rafael Capilla, España.\\
        \hline
        ... & ... \\
        \hline
    \end{tabular}
\end{table}

\subsubsection{CONFIGURADOR DE MAPA}

En la aplicación de Admin de Dropcontrol, en la vista del campo, se debe hacer click en la carta del mapa en el ícono de engranage para poder entrar al configurador de mapa.
En esta vista se divide en 2 secciones: Tabla de sectores/nodos y el mapa.

En la tabla de sectores/nodos se compone de:

\begin{itemize}
    \item \textbf{Botones:}
    \SubItem{En la parte superior izquierda estan los botónes para cambiar entre la tabla de sectores y nodos.}
    \subitem{En el lado opuesto, se encuentra el botón de "Cargar Archivo" que, como indica el nombre, es para poder subir el archivo .kmz o .kml que contiene los polígonos y puntos para asignar.}
    \item Tabla: La tabla muestra los sectores o nodos, según lo seleccionado. Contiene dos columnas, la primera con el nombre del sector/nodo y la segunda es una columnna de acciones, con un botón que sirve para centrar el sector/nodo en el mapa.
\end{itemize}

Y el mapa muestra los sectores y/o nodos que estaban previamente asignados. Si esta seleccionada la tabla de sectores en el mapa se muestran solo los polígonos, mismo caso con la tabla de nodos.

Al cargar un archivo, las secciones cambian. En la tabla de sectores/nodos ocurre lo siguiente:

\begin{itemize}
    \item El botón de "Cargar Archivo" desaparece y se reemplaza por el nombre del archivo cargado junto a un ícono de basura que al hacerle click, elimina el archivo y vuelve al estado anterior.
    \item En la tabla se agrega una columna nueva al medio, que contiene un dropdown con los polígonos o nodos (según la tabla seleccionada). Además, en la columna de acciones se agregan dos nuevas acciones: volver al polígono/punto previamente asignado y eliminar asignación. 
\end{itemize}

Respecto al mapa, se muestran los polígonos o puntos, según la tabla seleccionada, con un color distintivo.

\subsubsection{ALINEADOR DE IMÁGENES}

Esta funcionalidad se encuentra en la sección de proveedores externos en la herramienta de Admin de Dropcontrol. Para acceder 

\subsubsection{CONFIGURADOR DE MAPA}

Esta nueva funcionalidad se encuentra en una nueva sección "Graficador Libre" en la herramienta de Admin de Dropcontrol. Esta funcionalidad se divide en cuatro secciones:
\begin{itemize}
    \item \textbf{Tabla de nodos y sensores:} Aquí es donde se seleccionan los nodos con su sensor que se quieren desplegar en el gráfico. Se puede seleccionar un máximo de 6 filas. 
    \item \textbf{Selector de rango de fechas:} Seleccionar el rango de fecha, máximo de 90 días, de los datos.
    \item \textbf{Gráfico:} Aquí es donde se muestran los datos de los nodos y sensores.
    \item \textbf{Tabla de datos del gráfico:} Tabla de los datos de los sensores seleccionados. La primera columna corresponde a la fecha y las columnas siguientes corresponden a los sensores seleccionados.
\end{itemize}

\subsubsection{INTEGRADORES}

\subsubsection{CERTIFICADOS TESTBED}