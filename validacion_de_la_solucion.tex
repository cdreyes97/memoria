\secnumbersection{VALIDACIÓN DE LA SOLUCIÓN}

En la presente sección se presentarán las pruebas que validan los requerimientos de las herramientas/funcionalidades desarrolladas para las distintas aplicaciones de WiseConn.

\subsection{ADMIN DE DROPCONTROL}

\subsubsection{CONFIGURADOR DE MAPA}

\subsubsubsection{FUNCIONAMIENTO}

Para poder entrar a esta herramienta se debe dirigir al \textit{Dashboard de campo} de \textit{Admin}, y en la sección de mapa hacer click sobre el ícono que muestra la figura \ref{fig:mapcfg-1}.

\begin{figure}[H]
	\centering
	\includegraphics[width=0.8\textwidth]{mapcfg-1}
	\caption{\label{fig:mapcfg-1} Como ingresar al configurador de mapa}
\end{figure}

Al ingresar a esta herramienta, el usuario se encontrará con 2 secciones como se muestra en la figura \ref{fig:mapcfg-2}, teniendo a la izquierda la tabla de sectores y nodos y a la derecha el mapa.

\begin{figure}[H]
	\centering
	\includegraphics[width=0.8\textwidth]{mapcfg-2}
	\caption{\label{fig:mapcfg-2} Configurador de mapa}
\end{figure}

En la tabla, para cambiar entre sectores y nodos se debe hacer click en los botones de la esquina superior izquierda de la tabla.

\begin{figure}[H]
	\centering
	\includegraphics[width=0.8\textwidth]{mapcfg-tables}
	\caption{\label{fig:mapcfg-tables} Tabla de nodos y sectores}
\end{figure}

El mapa mostrará sectores o nodos según la tabla seleccionada como se ve en la figura \ref{fig:mapcfg-maps}. El primer mapa muestra los sectores representados por polígonos, mientras que, el segundo mapa muestra los nodos representados por marcadores. Como muestra en la figura, los sectores ya asignados son polígonos verdes y los nodos tienen un ícono según su tipo.

\begin{figure}[H]
	\centering
	\includegraphics[width=0.8\textwidth]{mapcfg-maps}
	\caption{\label{fig:mapcfg-maps} Tabla de nodos y sectores}
\end{figure}

Para cargar el archivo que contiene nuestros sectores y/o nodos, se debe hacer click sobre el botón de 'Cargar Archivo', como se muestra en la figura \ref{fig:mapcfg-load-file}

\begin{figure}[H]
	\centering
	\includegraphics[width=0.8\textwidth]{mapcfg-load-file}
	\caption{\label{fig:mapcfg-load-file} Carga de archivo .kmz}
\end{figure}

Al cargar el archivo .kmz, en las tablas se agrega una columna en medio con un selector como se aprecia en la figura \ref{fig:mapcfg-tables-edit}

\begin{figure}[H]
	\centering
	\includegraphics[width=0.8\textwidth]{mapcfg-tables-edit}
	\caption{\label{fig:mapcfg-tables-edit} Tablas de sectores y nodos al cargar un archivo.}
\end{figure}

En el mapa se agregan los polígonos del archivo con un color celeste (figura \ref{fig:mapcfg-cfg-map-zone}), mientras que los marcadores se muestran con un punto con un borde negro y centro azul (figura \ref{fig:mapcfg-cfg-map-node}).

\begin{figure}[H]
	\centering
	\includegraphics[width=0.8\textwidth]{mapcfg-cfg-map-zone}
	\caption{\label{fig:mapcfg-cfg-map-zone} Tablas de sectores y nodos al cargar un archivo.}
\end{figure}

\begin{figure}[H]
	\centering
	\includegraphics[width=0.8\textwidth]{mapcfg-cfg-map-node}
	\caption{\label{fig:mapcfg-cfg-map-node} Tablas de sectores y nodos al cargar un archivo.}
\end{figure}

\subsubsubsection{DATOS}

En el año 2023, como muestra la figura \ref{fig:mapcfg-analytics-view} el configurador de mapa tuvo 769 vistas, con un total de 187 usuarios y con una retención de aproximadamente 3 minutos.
\begin{figure}[H]
	\centering
	\includegraphics[width=0.8\textwidth]{mapcfg-analytics-view}
	\caption{\label{fig:mapcfg-analytics-view} Datos de vistas de la herramienta de configurador de mapa en el año 2023.}
\end{figure}

\subsubsection{GRAFICADOR LIBRE}

\subsubsubsection{FUNCIONAMIENTO}

Esta nueva herramienta implementada en la aplicación de \textit{Admin de DropControl} se encuentra en el menú lateral como muestra en la figura \ref{fig:menu-admin-graf}

\begin{figure}[H]
	\centering
	\includegraphics[width=0.5\textwidth]{menu-admin-graficador}
	\caption{\label{fig:menu-admin-graf} Graficador Libre en el menú de \textit{Admin}}
\end{figure}

En la herramienta se encontrará con 4 secciones, como se muestra en la figura \ref{fig:graf1}:
\begin{enumerate}
    \item Selección de nodo/sensores.
    \item Selección de rango de fechas.
    \item Gráfico.
    \item Tabla de datos.
\end{enumerate}

\begin{figure}[H]
	\centering
	\includegraphics[width=0.8\textwidth]{graficador-1}
	\caption{\label{fig:graf1} Secciones del graficador libre}
\end{figure}

El primer paso es escoger los sensores que queremos visualizar en el gráfico, en la primera sección selecciones en el primer selector el nodo seguido del sensor y opcionalmente el color (figura \ref{fig:grafselect}).

\begin{figure}[H]
	\centering
	\includegraphics[width=0.8\textwidth]{graf-select}
	\caption{\label{fig:grafselect} Selección de sensor}
\end{figure}

Para escoger otro sensor se puede hacer agregando otra fila vacía haciendo click en el botón 'Agregar' (figura \ref{fig:graf-add-row-1}) y repitiendo el paso anterior.

\begin{figure}[H]
	\centering
	\includegraphics[width=0.8\textwidth]{graf-add-row-1}
	\caption{\label{fig:graf-add-row-1} Agregar una nueva fila vacía}
\end{figure}

Otra forma de agregar sensores es con los botones con el símbolo '+' presente en las filas. El primer botón agrega una nueva fila con el siguiente nodo seleccionado junto al sensor del mismo tipo (figura \ref{fig:graf-add-row-2}), mientras que el segundo botón agrega una nueva fila con el mismo nodo seleccionado pero con el siguiente sensor en la lista (figura \ref{fig:graf-add-row-3}).

\begin{figure}[H]
	\centering
	\includegraphics[width=0.8\textwidth]{graf-add-row-2}
	\caption{\label{fig:graf-add-row-2} Agregar una nueva fila con el siguiente nodo}
\end{figure}

\begin{figure}[H]
	\centering
	\includegraphics[width=0.8\textwidth]{graf-add-row-3}
	\caption{\label{fig:graf-add-row-3} Agregar una nueva fila con el siguiente sensor del nodo}
\end{figure}

Para eliminar una fila se debe hacer click en el último botón de la fila, con símbolo de basura (figura \ref{fig:graf-delete-row}).

\begin{figure}[H]
	\centering
	\includegraphics[width=0.8\textwidth]{graf-delete-row}
	\caption{\label{fig:graf-delete-row} Agregar una nueva fila con el siguiente sensor del nodo}
\end{figure}

El siguiente paso es escoger el rango de fechas de los datos, como se muestra en la figura \ref{fig:graf-date-range-1}.

\begin{figure}[H]
	\centering
	\includegraphics[width=0.8\textwidth]{graf-date-range-1}
	\caption{\label{fig:graf-date-range-1} Selección de rango de fechas}
\end{figure}

Se puede escoger un rango personalizado, como se mostró anteriormente, o escoger rangos predeterminados como muestra en la figura \ref{fig:graf-date-range-2}.

\begin{figure}[H]
	\centering
	\includegraphics[width=0.8\textwidth]{graf-date-range-2}
	\caption{\label{fig:graf-date-range-2} Selección de rango de fechas}
\end{figure}

Luego de tener el rango de fechas seleccionado, hacer click en el botón 'ver' para mostrar los datos de los sensores seleccionados (figura \ref{fig:graf-display-1})

\begin{figure}[H]
	\centering
	\includegraphics[width=0.8\textwidth]{graf-display-1}
	\caption{\label{fig:graf-display-1} Gráfico}
\end{figure}

En la última sección se encuentra la tabla de datos del gráfico, en donde se muestran los datos de los sensores donde la primera columna es la fecha y hora del dato y las columnas siguientes son los sensores escogidos con el formato de nombre \{Nodo\}-\{Sensor\} [unidad] (figura \ref{fig:graf-table-1}). 
Además, tiene la funcionalidad de exportar la tabla en formato .xlsx haciendo click en el ícono de descarga en la esquina superior derecha de la tabla.

\begin{figure}[H]
	\centering
	\includegraphics[width=0.8\textwidth]{graf-table-1}
	\caption{\label{fig:graf-table-1} Gráfico}
\end{figure}


\subsubsection{DATOS}

En el año 2023, como muestra la figura \ref{fig:graf-analytics-view} el configurador de mapa tuvo 2947 vistas, con un total de 533 usuarios y con una retención de aproximadamente 1 minuto.
\begin{figure}[H]
	\centering
	\includegraphics[width=0.8\textwidth]{graf-analytics-view}
	\caption{\label{fig:graf-analytics-view} Datos de vistas de la herramienta de graficador libre en el año 2023.}
\end{figure}


\subsection{OPERATIONS}

\subsubsection{DESPACHOS MÚLTIPLES}

Entrando a la herramienta de \textit{Operations}, en el menú lateral, abrir la sección de Despachos y se encuentran las opciones de 'Individuales' y 'Múltiples' (Figura \ref{fig:op-menu}).

\begin{figure}[H]
	\centering
	\includegraphics[width=1\textwidth]{validation-op-dm/menu.png}
	\caption{\label{fig:op-menu} Menu de Despachos en \textit{Operations}}
\end{figure}

Al hacer click en 'Múltiples' se ingresa a la sección de despachos múltiples (Figura \ref{fig:op-list}).
Se tiene una tabla con los despachos múltiples creados.

\begin{figure}[H]
	\centering
	\includegraphics[width=1\textwidth]{validation-op-dm/list.png}
	\caption{\label{fig:op-list} Menu de Despachos en \textit{Operations}}
\end{figure}

Para crear un despacho múltiple se debe hacer click en el botón 'Crear' en la tabla. Al hacer click, se abrirá el formulario de la figura \ref{fig:op-form-create-1}.
Los parámetros previos al 'Número de despachos' son los parámetros del despacho padre, el cual los despachos hijos herederan.

\begin{figure}[H]
	\centering
	\includegraphics[width=1\textwidth]{validation-op-dm/form-create-1.png.png}
	\caption{\label{fig:op-form-create-1} Formulario de creación de Despacho Múltiple}
\end{figure}

El parámetros 'Número de despachos' indica, valga la redundancia, el número de despachos hijos. Esto se aprecia al comparar la figura \ref{fig:op-form-create-1} y la figura \ref{fig:op-form-create-2}.

\begin{figure}[H]
	\centering
	\includegraphics[width=1\textwidth]{validation-op-dm/form-create-2.png.png}
	\caption{\label{fig:op-form-create-2} Formulario de creación de Despacho Múltiple con 3 despachos}
\end{figure}

\subsection{SETUP}

\subsubsection{CONFIGURADOR DE FUENTES}

En el configurador \textit{Wizard} de \textit{Setup}, al llegar al paso 'Opciones' se encuentran distintas configuraciones que se pueden realizar, dentro de estas se encuentra 'Configuración de Fuentas' como muestra en la figura \ref{fig:ws-wizard-option}

\begin{figure}[H]
	\centering
	\includegraphics[width=1\textwidth]{validation-watersources/wizard-option.png}
	\caption{\label{fig:ws-wizard-option} Configuración de Fuentes en el asistente de configuración \textit{Wizard}}
\end{figure}

Al seleccionar la opción de 'Configuración de Fuentes' y hacer click en siguiente, se redirige a la herramienta.

\begin{figure}[H]
	\centering
	\includegraphics[width=0.8\textwidth]{water-view}
	\caption{\label{fig:water-view} Configurador de fuentes. Fuente: Elaboración propia.}
\end{figure}

A continuación se mostrarán las conexiones permitidas según muestra la figura \ref{fig:water-connections}:

\begin{figure}[H]
	\centering
	\includegraphics[width=0.8\textwidth]{validation-watersources/from-canal.png}
	\caption{\label{fig:from-canal} Conexiones desde un canal. Fuente: Elaboración propia.}
\end{figure}

\begin{figure}[H]
	\centering
	\includegraphics[width=0.8\textwidth]{validation-watersources/from-pozo.png}
	\caption{\label{fig:from-pozo} Conexiones desde un pozo. Fuente: Elaboración propia.}
\end{figure}

\begin{figure}[H]
	\centering
	\includegraphics[width=0.8\textwidth]{validation-watersources/from-tranque.png}
	\caption{\label{fig:from-tranque} Conexiones desde un tranque. Fuente: Elaboración propia.}
\end{figure}

\begin{figure}[H]
	\centering
	\includegraphics[width=0.8\textwidth]{validation-watersources/from-impulsion.png}
	\caption{\label{fig:from-impulsion} Conexiones desde una impulsión. Fuente: Elaboración propia.}
\end{figure}

\begin{figure}[H]
	\centering
	\includegraphics[width=0.8\textwidth]{validation-watersources/from-equipo.png}
	\caption{\label{fig:from-equipo} Conexiones desde un equipo de riego. Fuente: Elaboración propia.}
\end{figure}