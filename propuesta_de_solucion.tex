\secnumbersection{PROPUESTA DE SOLUCIÓN}

\subsection{DISEÑO DE LA SOLUCIÓN}

Entendiendo los problemas de los productos de software quue ofrece la empresa y las consecuencias que esto conlleva,
se plantean nuevas herramientas de software y/o mejoras a herramientas existentes que ayuden a liberar carga y esfuerzo
de ciertos procesos a las áreas de soporte y producción, además, darle al usuario cliente nuevas herramientas para que tenga un mejor flujo de trabajo.

\subsubsection{CONFIGURADOR DE MAPA}

En la aplicación \textit{DropControl}, el usuario tiene a su disposición el mapa de su campo (Agregar imagen). 
Para mostrar el mapa se utiliza \textit{Google Maps} y se utilizan los polígonos para marcar los sectores del campo y los puntos para los nodos.
Para poder configurar este mapa se puede hacer de dos maneras:
\begin{enumerate}
    \item En la aplicación \textit{Admin de DropControl}, ingresar a cada sector/nodo y agregar su respectivo polígono/punto de manera individual.
    \item Enviar un ticket al área de soporte con un archivo con formato .kmz o .kml que contiene los polígonos y/o puntos para sus respectivos sectores y/o nodos.
\end{enumerate}
Para la primera opción, si el campo tiene demasiados sectores/nodos conllevará mucho tiempo en asginar los polígonos/puntos respectivos.
En la segunda opción existen problemas como el tiempo que demora el área tome el ticket y realize la confgiruación y/o 
que las instrucciones no están bien indicadas llevando a que esta configuración requiera varios tickets para que se haga como el usuario cliente quiera.

Para esto se plantea una herramienta en la aplicación de \textit{Admin de DropControl} para que el usuario clienta sea el que
configure el mapa, subiendo su archivo .kmz o .kml y asignando los polígonos/puntos a sus respectivos sectores/nodos.

\subsubsection{ALINEADOR DE IMÁGENES}

Una de las herramientas que ofrece \textit{DropControl} es el análisis de imágenes utilizando 
el índice de vegetación de diferencia normalizada o NDVI por sus siglas en inglés. Para obtener estas imágenes satelitales
se utilizan otros proveedores y se muestran en \textit{DropControl} con \textit{Google Maps}, pero no todas las imágenes tienen
las mismas coordenadas en \textit{Google Maps} por lo que se verán desalineadas. Esto se hace mandando un ticket al área de soporte
y son ellos lo que se encargan de alinear la imagen.

La solución para este problema se plantea una nueva herramienta para la aplicación de \textit{Admin de DropControl} en la que 
se puedan alinear las imágenes satelitales a \textit{Google Maps}, para esto se coloca la imagen sobre el mapa y es el usuario cliente
el que debe alinear la imagen.

\subsection{EJEMPLO DE COMO CITAR FIGURAS E ILUSTRACIONES}

Se colocó una imagen que se puede referenciar también desde el texto (Ver figura \ref{fig:malla}).

\begin{figure}[h]
\centering
\includegraphics[width=0.8\textwidth]{malla_ingenieria_informatica}
\caption{\label{fig:malla} Malla Curricular Ingeniería Civil Informática.} Fuente: Departamento de Informática.
\end{figure}
