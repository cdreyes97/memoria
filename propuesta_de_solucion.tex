\secnumbersection{PROPUESTA DE SOLUCIÓN}

\subsection{DISEÑO DE LAS SOLUCIÓN}

Entendiendo los problemas de los productos de software quue ofrece la empresa y las consecuencias que esto conlleva,
se plantean nuevas herramientas de software y/o mejoras a herramientas existentes que ayuden a liberar carga y esfuerzo
de ciertos procesos a las áreas de soporte y producción, además, darle al usuario cliente nuevas herramientas para que tenga un mejor flujo de trabajo.

\subsubsection{CONFIGURADOR DE MAPA}

En la aplicación \textit{DropControl}, el usuario tiene a su disposición el mapa de su campo (Agregar imagen). 
Para mostrar el mapa se utiliza \textit{Google Maps} y se utilizan los polígonos para marcar los sectores del campo y los puntos para los nodos.
Para poder configurar este mapa se puede hacer de dos maneras:
\begin{enumerate}
    \item En la aplicación \textit{Admin de DropControl}, ingresar a cada sector/nodo y agregar su respectivo polígono/punto de manera individual.
    \item Enviar un ticket al área de soporte con un archivo con formato .kmz o .kml que contiene los polígonos y/o puntos para sus respectivos sectores y/o nodos.
\end{enumerate}
Para la primera opción, si el campo tiene demasiados sectores/nodos conllevará mucho tiempo en asginar los polígonos/puntos respectivos.
En la segunda opción existen problemas como el tiempo que demora el área tome el ticket y realize la confgiruación y/o 
que las instrucciones no están bien indicadas llevando a que esta configuración requiera varios tickets para que se haga como el usuario cliente quiera.

Para esto se plantea una herramienta en la aplicación de \textit{Admin de DropControl} para que el usuario clienta sea el que
configure el mapa, subiendo su archivo .kmz o .kml y asignando los polígonos/puntos a sus respectivos sectores/nodos.
Junto con esto, tambien se agrega la funcionalidad de poder descargar el mapa de su campo en formato .kmz.

\subsubsection{ALINEADOR DE IMÁGENES}

Una de las herramientas que ofrece \textit{DropControl} es el análisis de imágenes utilizando 
el índice de vegetación de diferencia normalizada o NDVI\footnote{\href{https://eos.com/es/make-an-analysis/ndvi/}{NDVI: Índice De Vegetación De Diferencia Normalizada}} por sus siglas en inglés. Para obtener estas imágenes satelitales
se utilizan otros proveedores y se muestran en \textit{DropControl} con \textit{Google Maps}, pero no todas las imágenes tienen
las mismas coordenadas en \textit{Google Maps} por lo que se verán desalineadas. Esto se hace mandando un ticket al área de soporte
y son ellos lo que se encargan de alinear la imagen.

La solución para este problema se plantea una nueva herramienta para la aplicación de \textit{Admin de DropControl} en la que 
se puedan alinear las imágenes satelitales a \textit{Google Maps}, para esto se coloca la imagen sobre el mapa y es el usuario cliente
el que debe alinear la imagen.

\subsubsection{GRAFICADOR LIBRE}

Como se explicó anteriormente, el plan gratuito que ofrece \textit{Wiseconn} sirve como punto de entrada a las herramientas
de pago de \textit{DropControl}. Sin embargo, el plan gratuito ofrece muy pocas funcionalidades y/o herramientas
implicando poca retención de los usuarios y no se cumplen con las especificaciones comerciales.

Los usuarios clientes de plan de pago tienen a su disposición un graficador en la aplicación de \textit{DropControl}, en el cual
se pueden graficar los datos enviados por los nodos en los campos dentro de un rango de tiempo.
Caso contrario ocurre para los usuarios clientes de plan gratuito, los cuales no poseen esta herramienta y 
no pueden visualizar los datos de sus nodos.

Por esto se desarrolla la herramienta de "Graficado Libre" en la aplicación de \textit{Admin de DropControl}, 
donde el usuario podrá escoger hasta 6 nodos para graficar en un rango de tiempo máximo de 3 meses (90 días).

\subsubsection{MARKETPLACE INTEGRADORES}



\subsubsection{CONFIGURADOR DE FUENTES}



\subsubsection{CERTIFICADOS TESTBED}

Dentro de las tareas que cumple el área de producción es son las pruebas del hardware, cada prueba queda documentada con
un certificado en donde se indica si el hardware pasó o no las pruebas para seguir con su producción. Estos certificados son
almacenados en un servidor FTP que después se guardan en un bucket de S3.

La aplicación de \textit{Operations} es para la gestión de lotes y despachos de productos, además de la edición de productos. 
El producto tiene un historial con historias asociadas que indican ingresos a lotes, despachos, marcar como producto fallado y/o reparado.

Los trabajadores para acceder a los certificados testbed...

Teniendo esto en cuenta, se plantea implementar una historia de certificados que se agreguen automáticamente cuando se guarde un
certificado.

\subsection{IMPLEMENTACIÓN DE LA SOLUCIÓN}

\subsubsection{CONFIGURADOR DE MAPA}
\subsubsection{ALINEADOR DE IMÁGENES}
\subsubsection{GRAFICADOR LIBRE}
\subsubsection{MARKETPLACE INTEGRADORES}
\subsubsection{CONFIGURADOR DE FUENTES}
\subsubsection{CERTIFICADOS TESTBED}


\subsection{EJEMPLO DE COMO CITAR FIGURAS E ILUSTRACIONES}

Se colocó una imagen que se puede referenciar también desde el texto (Ver figura \ref{fig:malla}).

\begin{figure}[h]
\centering
\includegraphics[width=0.8\textwidth]{malla_ingenieria_informatica}
\caption{\label{fig:malla} Malla Curricular Ingeniería Civil Informática.} Fuente: Departamento de Informática.
\end{figure}
